\section{Ultrafiltration and diafiltration modelling}

\subsection{Poisson-boltzmann equation}

\subsubsection{Problem definition}

Within the water solution and where large, charged and spherical particles are present electrical
field formed around theme affects movement of the ions. This significantly impacts distribution
of ions within the aqueous solution and consequently final concentration of individual ions.
The discrepancy in the ion concentration further results in observable pH shift.

Monoclonal antibodies, nevertheless they are Y shaped, can, in first approximation be treated
as uniformly charged spheres of specific radius.

There are two main mechanisms:
\begin{itemize}
    \item steric displacement - impacts all ions,
    \item concentration variation due to the protein's electrical field - impacts only charged species.
\end{itemize}

Both mechanisms are schematically depicted in figure (ADD FIGURE).

In order to estimate impact of electrical field on the concentration of ions first step is to
evaluate the potential generated by the individual protein.

Maxwell equations govern how electro-magnetic fields are formed. In particular, relevant for
protein's EDL is following Maxwell equation:

\begin{equation}
    \nabla \ve{E(\ve{r})} = \frac{\rho(\ve{r})}{\epsilon_r\epsilon_0},
\end{equation}

where $\ve{E}(\ve{r})$ is electrical field generated by charge density $\rho(\ve{r})$.
$\epsilon_0$ is a vacuum permittivity. It is assumed that the electrical field is formed in
uniform continuos medium, the solvent (in our case water). Thus, additional
correction is required in the form of dimensionless relative permittivity $\epsilon_r$.
Often both are combined into water dielectric constant $D= \epsilon_r\epsilon_0$.

\begin{equation}
    \ve{E}(\ve{r}) = -\nabla \psi(\ve{r})
\end{equation}

Combining both equations:
\begin{equation}\label{eq:PoissonInMatter1}
    \nabla^2 \psi(\ve{r}) = \frac{\rho(\ve{r})}{\epsilon_r\epsilon_0}
\end{equation}

To calculate the charge density we assume following:
\begin{itemize}
    \item charges are carried around by the ions present in water,
    \item protein charge is calculated based on the pH of the solution,
    \item pH of the solution depends on all present charged species (including protein),
    \item number density distribution is governed by the Boltzmann distribution.
\end{itemize}

Boltzmann distribution tells us that

\begin{equation}
    \rho_{ji}(\ve{r}) \propto \rho_{ji}^{\infty} \exp\left(\frac{q_{ji}\psi(\ve{r})}{k_BT}\right),
\end{equation}

where $j$ counts acid components $j \in [1, K]$ and $i$ counts the acid's dissociated components
(ions), $i \in [0, n_j]$.

Hence, overall charge at position $\ve{r}$ is sum of above contributions:
\begin{equation}\label{eq:BoltzmannDist1}
    \rho(\ve{r}) = \sum_{j = 1}^{K}\sum_{i=0}^{n_j} \rho_{ji}(\ve{r}) =
    \sum_{j = 1}^{K}\sum_{i=0}^{n_j} \rho_{ji}^{\infty} \exp\left(\frac{q_{ji}\psi(\ve{r})}{k_BT}\right),
\end{equation}

Explanation of the quantities:
\begin{itemize}
    \item $\rho_{ji}(\ve{r})$ - particle number density of $ji$-th ion component in water solution.
    \item $\rho_{ji}^{\infty}$ - particle number density of $ji$-th ion component far away from the protein.
    \item $q_ji$ - charge of $ji$-th ion component.
    \item $T$ - absolute temperature.
    \item $k_B$ - Boltzmann constant.
\end{itemize}

Let us assume that electrical field around the protein is isotropic (independent of direction). Therefore,
it only depends on the distance from the protein's center, which means $\psi(\vec{r}) = \psi(r)$.

In order to exploit this symmetry it is convenient to write out the Laplace operator in spherical coordinates \footnote{for derivation see APPENDIX}
\begin{equation} \label{eq:lapaceInSphericalCoords1}
    \nabla^2 \equiv \frac{1}{r^2}\frac{\partial}{\partial r} \left(r^2\frac{\partial}{\partial r}\right)
    + \frac{1}{r^2 \sin^2\phi} \frac{\partial^2}{\partial\theta^2}
    + \frac{1}{r^2 \sin \phi} \frac{\partial}{\partial \phi}\left(\sin\phi \frac{\partial}{\partial\phi}\right).
\end{equation}

As \psi only depends on the distance $r$ and not




\paragraph*{Dimensionless quantities}


\[Y = \frac{e_0 \psi}{k_B T}\]
\[x = \frac{r}{a}\]
\[a = \sqrt{\frac{\epsilon_r\epsilon_0k_BT}{N_Ae_0^22I}} \]



\subsubsection{High temperature limit}

\begin{equation}\label{eq:pbHighTempDimensionless}
    \frac{\ud^2Y}{\ud x^2} + \frac{2}{x} \frac{\ud Y}{\ud x} - Y = 0
\end{equation}

Boundary conditions:

\subsubsection{Debye length}

\subsection{Tangential flow filtration model}